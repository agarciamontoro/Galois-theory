\documentclass[a4paper, 11pt]{article}
\usepackage[semana=5]{estilo}

\begin{document}

  \maketitle

  \begin{ejercicio}
      Calcula el grado de los cuerpos de descomposición de los siguientes polinomios sobre $\mathbb{Q}$:
      \begin{enumerate}
          \item $p(X) = X^3-2$
          \item $q(X) = X^4-7$
          \item $r(X) = (X^2-2)(X^2-5)$
      \end{enumerate}
  \end{ejercicio}

  \begin{apartado}
  \end{apartado}

  \begin{solucion}
      Para obtener el cuerpo de descomposición de un polinomio sobre un cuerpo $K$, basta sacar sus raíces $\alpha_1,\dots,\alpha_n$ y construir la extensión $K(\alpha_1,\dots,\alpha_n)$.

      Las raíces de $p(X)$ son tres: $\sqrt[3]{2},\omega\sqrt[3]{2}$ y $\omega^2\sqrt[3]{2}$, donde $\omega$ es una raíz tercera de la unidad; es decir, es un elemento tal que $\omega^3=1$.

      Por tanto, el cuerpo de descomposición de $p(X)$ es $\mathbb{Q}(\sqrt[3]{2},\omega\sqrt[3]{2},\omega^2\sqrt[3]{2})=\mathbb{Q}(\omega,\sqrt[3]{2})$.

      Sabemos que
      \[
      [\mathbb{Q}(\omega,\sqrt[3]{2}):\mathbb{Q}] = [\mathbb{Q}(\omega,\sqrt[3]{2}):\mathbb{Q}(\sqrt[3]{2})][\mathbb{Q}(\sqrt[3]{2}):\mathbb{Q}]
      \]

      Como $\sqrt[3]{2}$ es raíz de $p(X)$, polinomio irreducible ---por el criterio de Eisenstein, por ejemplo--- de grado 3, tenemos que $[\mathbb{Q}(\sqrt[3]{2}):\mathbb{Q}] = 3$.

      Además, sabemos que podemos acotar el grado del cuerpo de descomposición por el factorial del grado del polinomio:
      \[
      [\mathbb{Q}(\omega,\sqrt[3]{2}):\mathbb{Q}] \leq 3! = 6
      \]

      Por tanto, el grado que buscamos cumple:
      \begin{itemize}
          \item Es un múltiplo de 3.
          \item Es menor o igual que 6.
      \end{itemize}

      Tenemos entonces que el grado o bien es 3 o bien es 6.

      Supongamos que es 3. Entonces, $[\mathbb{Q}(\omega,\sqrt[3]{2}):\mathbb{Q}(\sqrt[3]{2})] = 1$. Esto implica que  $\mathbb{Q}(\omega,\sqrt[3]{2}) = \mathbb{Q}(\sqrt[3]{2})$. Pero sabemos que $\omega\notin\mathbb{Q}(\sqrt[3]{2})$, así que $\mathbb{Q}(\omega,\sqrt[3]{2}) \neq \mathbb{Q}(\sqrt[3]{2})$. Esto es una contradicción con la suposición de que el grado es 3.

      Por tanto, concluimos que
      \[
      [\mathbb{Q}(\omega,\sqrt[3]{2}):\mathbb{Q}] = 6
      \]
  \end{solucion}

  \begin{apartado}
  \end{apartado}

  \begin{solucion}
      Las raíces de $q(X)$ son $\sqrt[4]{7},\xi\sqrt[4]{7},\xi^2\sqrt[4]{7}$ y $\xi^3\sqrt[4]{7}$, donde $\xi$ es una raíz cuarta de la unidad; esto es, un elemento tal que $\xi^4=1$.

      Entonces, tenemos que el cuerpo de descomposición de $q(X)$ es
      \[
      \mathbb{Q}(\sqrt[4]{7},\xi\sqrt[4]{7},\xi^2\sqrt[4]{7},\xi^3\sqrt[4]{7}) = \mathbb{Q}(i,\sqrt[4]{7})
      \]
      donde $i=\xi$ es la notación usual de la raíz cuarta de la unidad.

      Igual que antes, podemos deducir que $[\mathbb{Q}(\sqrt[4]{7}):\mathbb{Q}] = 4$. Además, como $[\mathbb{Q}(i,\sqrt[4]{7}):\mathbb{Q}] \leq 4! = 24$, tenemos que $[\mathbb{Q}(i,\sqrt[4]{7}):\mathbb{Q}] \in \{4,8,12,24\}$.

      Es claro que $[\mathbb{Q}(i,\sqrt[4]{7}):\mathbb{Q}] > 4$, pues si fuera igual llegaríamos a la misma contradicción de antes ---porque $i\notin\mathbb{Q}(\sqrt[4]{7})$---.

      Además, como $i$ es raíz del polinomio $X^4-1\in\mathbb{Q}(\sqrt[4]{7})$, sabemos que $[\mathbb{Q}(i,\sqrt[4]{7}):\mathbb{Q}(\sqrt[4]{7})]\leq 2$.

      Concluimos por tanto que
      \[
      [\mathbb{Q}(i,\sqrt[4]{7}):\mathbb{Q}] = 8
      \]
  \end{solucion}

  \begin{apartado}
  \end{apartado}

  \begin{solucion}
      El cuerpo de descomposición de $r(X)$ es $\mathbb{Q}(\sqrt{2},\sqrt{5})$.

      El grado es
      \[
      [\mathbb{Q}(\sqrt{2},\sqrt{5}):\mathbb{Q}] = [\mathbb{Q}(\sqrt{2},\sqrt{5}):\mathbb{Q}(\sqrt{2})][\mathbb{Q}(\sqrt{2}):\mathbb{Q}]
      \]

      Sabemos que $[\mathbb{Q}(\sqrt{2}):\mathbb{Q}] = 2$ por ser $\sqrt{2}$ una raíz del polinomio irreducible $X^2-2$. Además, como $\sqrt{5}\notin\mathbb{Q}(\sqrt{2})$, tenemos que $[\mathbb{Q}(\sqrt{2},\sqrt{5}):\mathbb{Q}(\sqrt{2})] = 2$.

      Concluimos así que
      \[
      [\mathbb{Q}(\sqrt{2},\sqrt{5}):\mathbb{Q}] = 4
      \]

  \end{solucion}

  \newpage
  \begin{ejercicio}
      ¿Es la extensión $\mathbb{Q}\left(\sqrt{2+\sqrt{-5}}\right)/\mathbb{Q}$ normal?
  \end{ejercicio}

  \begin{solucion}
      Para ver si la extensión indicada es normal, basta ver si es cuerpo de descomposición de algún polinomio.

      Para esto, estudiemos el elemento $\alpha = \sqrt{2+\sqrt{-5}}$. Manipulando este elemento, podemos ver de qué polinomio sobre $\mathbb{Q}$ es raíz:

      \begin{align*}
          \alpha = \sqrt{2+\sqrt{-5}} \\
          \comment{Elevando al cuadrado} \alpha^2 = 2+\sqrt{-5} \\
          \alpha^2 - 2 = \sqrt{-5} \\
          \comment{Elevando al cuadrado} \alpha^4 - 4\alpha^2 + 4 = -5 \\
          \alpha^4 - 4\alpha^2 + 9 = 0
      \end{align*}
  \end{solucion}

  Es claro entonces que $\alpha$ es raíz del polinomio $p(X)=X^4-4X^2+9\in\mathbb{Q}[X]$.

  Podemos calcular entonces el cuerpo de descomposición de $p(X)$:

  Sus raíces ---que obtenemos, por ejemplo, haciendo el cambio de variable $Y=X^2$ y resolviendo--- son $\pm\sqrt{2+\sqrt{-5}}$ y $\pm\sqrt{2-\sqrt{-5}}$.

  Por tanto, el cuerpo de descomposición de $p(X)$ es
  \[
  \mathbb{Q}(\sqrt{2+\sqrt{-5}},\sqrt{2-\sqrt{-5}})
  \]

  Podemos ahora estudiar si este cuerpo, en vez de generarlo con los dos radicales, podemos generarlo con uno. Esto es, ahora nos preguntamos:
  \[
  \mbox{¿} \mathbb{Q}(\sqrt{2+\sqrt{-5}},\sqrt{2-\sqrt{-5}}) = \mathbb{Q}(\sqrt{2+\sqrt{-5}}) ?
  \]

  Si la respuesta es afirmativa, habremos terminado el ejercicio, pues quedará probado que la extensión indicada en el enunciado es el cuerpo de descomposición del polinomio $p(X)=X^4-4X^2+9$ y será, por lo tanto, normal. Si la respuesta fuera negativa tendríamos que seguir buscando, pues esto no probaría que la extensión dada fuera no normal.

  Para ver si los dos cuerpos son iguales, basta ver que
  \[
  \mathbb{Q}(\sqrt{2+\sqrt{-5}},\sqrt{2-\sqrt{-5}}) \subset \mathbb{Q}(\sqrt{2+\sqrt{-5}})
  \]
  ---la otra inclusión es trivial---.

  Es decir, hay que probar que $\sqrt{2-\sqrt{-5}}\in\mathbb{Q}(\sqrt{2+\sqrt{-5}})$. Veamos qué pasa si multiplicamos los dos radicandos y despejamos nuestro elemento:
  \begin{align*}
      \sqrt{2+\sqrt{-5}}\sqrt{2-\sqrt{-5}} = \sqrt{4+5} = 3 \\
      \sqrt{2-\sqrt{-5}} = \frac{3}{\sqrt{2+\sqrt{-5}}}
  \end{align*}

  Como vemos, se puede escribir $\sqrt{2-\sqrt{-5}}$ como una fracción de elementos de $\mathbb{Q}(\sqrt{2+\sqrt{-5}})$; es decir, tenemos que
  \[
  \sqrt{2-\sqrt{-5}}\in\mathbb{Q}(\sqrt{2+\sqrt{-5}})
  \]

  Concluimos por tanto que
  \[
  \mathbb{Q}(\sqrt{2+\sqrt{-5}},\sqrt{2-\sqrt{-5}}) = \mathbb{Q}(\sqrt{2+\sqrt{-5}})
  \]
  y, así, queda probado que $\mathbb{Q}(\sqrt{2+\sqrt{-5}})$ es una extensión normal.
\end{document}
