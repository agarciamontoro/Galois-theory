\documentclass[a4paper, 11pt]{article}
\usepackage[semana=9]{estilo}

\begin{document}

  \maketitle
  \begin{ejercicio}
      Sea $q$ una potencia de un entero primo positivo.
      \begin{enumerate}
          \item Demostrar que un polinomio irreducible de grado $r$ sobre $\mathbb{F}_q$ es un factor de $X^{q^n}-X$ si, y sólo si, $r|n$.
          \item Deducir que
          \[
          X^{q^n}-X = \prod_i f_i(X)
          \]
          donde $f_i$ varía sobre todos los polinomios irreducibles cuyo grado divide a $n$.
          \item Demostrar que si $t_r$ es el número de tales polinomios, entonces
          \[
          \sum r t_r = q^n
          \]
          y deducir una fórmula para $t_r$, en términos de $q$, $r$ y la función de Möbius.
      \end{enumerate}
  \end{ejercicio}

  \begin{solucion}
      \begin{apartado}
          Llamamos $P(X)=X^{q^n}-X$.

          Sea $K$ un cuerpo con $q^n$ elementos. Sabemos que, si $\alpha\in\mathbb{K}^x$, entonces $\alpha^{q^n-1} = \alpha$. Es claro entonces que todos los elementos de $K$ son raíces de $P(X)$; de hecho, \emph{todas} las raíces del polinomio están en $K$, pues no puede tener más de $q^n$ raíces.

          Sea ahora $Q(X)\in\mathbb{F}_q$ un factor irreducible de $P(X)$. Como todas las raíces de $P(X)$ están en $K$, también lo están las de $Q$. Sea entonces $\beta\in K$ una raíz de $Q$.

          Tenemos entonces la torre de cuerpos
          \[
          \mathbb{F}_q \subset \mathbb{F}_q(\beta) \subset K
          \]
          donde $[\mathbb{F}_q(\beta):\mathbb{F}_q]$ es el grado del polinomio $Q$. Es claro entonces que el grado de $Q$ es un divisor de $n$.

          Sea ahora $Q(X)$ un irreducible de grado $r$, divisor de $n$. Usando el corolario 12.5 y teniendo en cuenta que la característica de $\mathbb{F}_q$ es $p$, es claro que $Q(X)$ es factor de $P(X)$.
      \end{apartado}

      \begin{apartado}
          Este resultado se deduce directamente de $1$.

          Todos los factores de $P(X)$ tienen grado divisor de $n$ y cualquier irreducible con grado divisior de $n$ es factor, luego el conjunto de todos los factores de $P(X)$ es exactamente igual al de todos los irreducibles de grado divisor de $n$; es decir:
          \[
          X^{q^n}-X = \prod_i f_i(X)
          \]
          donde $f_i$ varía sobre todos los polinomios irreducibles cuyo grado divide a $n$.
      \end{apartado}

      \begin{apartado}
          Dada la finitud de elementos en $\mathbb{F}_q$, este resultado es claro contando las raíces de $P(X)$.

          Sabemos que $P(X)$ tiene $q^n$ raíces por ser este el grado del polinomio. Para cada $r$ divisor de $n$, los irreducibles de las raíces con ese grado son factores de $P(X)$ y, por tanto, hay $rt_r$ de ellos.

          La suma de todos nos da la igualdad buscada:
          \[
          \sum_{r\in Div(n)} r t_r = q^n
          \]

          De la definición de la definición de Möbius,
          \[
          \mu(n) = \begin{cases}
              1 & \textrm{si $n=1$}\\
              (-1)^r & \textrm{si $n$ es el producto de $r$ enteros primos distintos}\\
              0 & \textrm{si $n$ es divisible por el cuadrado de un entero primo}
          \end{cases}
          \]
          y del resultado que acabamos de ver, es directo deducir la fórmula para $t_r$:
          \[
          t_r = \frac{1}{r}\sum_{d\in Div(r)}\mu(d)q^\frac{r}{d}
          \]
      \end{apartado}
  \end{solucion}

\end{document}
