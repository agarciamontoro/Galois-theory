\documentclass[a4paper, 11pt]{article}
\usepackage[semana=8]{estilo}

\begin{document}

  \maketitle
  \begin{ejercicio}
      Sea \(K\) un cuerpo y sean \(u\) y \(v\) dos elementos tales que las extensiones \(K(u)\) y \(K(v)\) tienen grados coprimos \(m\) y \(n\), respectivamente. ¿Cuál es el grado de \([K(u,v):K]\)?
  \end{ejercicio}

  \begin{solucion}
      Como tenemos la torre de cuerpos \(K \subset K(u) \subset K(u,v)\) ---ver figura \ref{subcuerpos}---, sabemos que
      \[
      [K(u,v):K]=[K(u,v):K(u)][K(u):K]=[K(u,v):K(u)]m
      \]
      luego \(m\mid[K(u,v):K]\).

      Lo mismo podemos decir de \(K \subset K(v) \subset K(u,v)\), así que \(n\mid[K(u,v):K]\).

      \begin{figure}[ht]
          \centering
          \begin{tikzpicture}[node distance=2cm]
              \node(GalEQ)                                          {$K(u,v)$};

              \node(GalEalpha)  [below left=1cm and 0.5cm  of GalEQ]    {$K(u)$};
              \node(GalEi)      [below right=1cm and 0.5cm of GalEQ]    {$K(v)$};

              \node(uno)        [below=2.5cm of GalEQ]                  {$K$};

              \draw(GalEQ)      -- (GalEalpha);
              \draw(GalEQ)      -- (GalEi);

              \draw(GalEalpha) -- (uno);
              \draw(GalEi)     -- (uno);
          \end{tikzpicture}
          \caption{Subcuerpos de $K(u,v)$.}
          \label{subcuerpos}
      \end{figure}

      De los dos hechos anteriores y teniendo en cuenta que $m$ y $n$ son coprimos, podemos concluir que \(mn\mid[K(u,v):K]\).

      Pero, además, sabemos que podemos acotar el grado que buscamos con el grado de las subextensiones de la siguiente manera:
      \[
      [K(u,v):K] \leq mn
      \]

      Es entonces evidente que
      \[
      [K(u,v):K]=mn
      \]
  \end{solucion}

\end{document}
