\documentclass[a4paper, 11pt]{article}
\usepackage[semana=6]{estilo}

\begin{document}

  \maketitle
  \begin{ejercicio}
      Se considera el grupo $C_p \times C_p$, siendo $p$ un entero primo positivo.
  \end{ejercicio}

  \begin{apartado}
      Prueba que $Aut(C_p \times C_p)$ es el grupo lineal general $GL(p,2)$.
  \end{apartado}

  \begin{solucion}
      \emph{Nota: } $GL(p,2) =$ matrices invertibles de orden 2 con sus elementos en $C_p$.

      En general, sabemos que si $K$ es un cuerpo y $V$ un espacio vectorial de dimensión $n$ sobre $K$, se tiene que:
      \[
      End_K(V) \cong \mathbb{M}_n(K)
      \]

      Además, si un endomorfismo es invertible ---es un automorfismo---, su matriz asociada es también invertible, con su inversa la matriz asociada al endomorfismo inverso. Esto es:
      \[
      Aut_K(V) \cong GL(p,n)
      \]

      En nuestro caso, tomando $K=C_p$ y $V=C_p \times C_p$ ---espacio vectorial de dimensión 2 sobre $C_p$---, se tiene el resultado pedido.

      Podemos demostrar el resultado general para espacios vectoriales de dimensión 2, como $C_p \times C_p$. Es decir, dado $K$ un cuerpo, veamos la biyección existente entre automorfismos de $K \times K$ y las matrices de dimensión 2 sobre el cuerpo $K$:

      Sea $f \in End(K \times K)$ y $B=\{e_1,e_2\}$ una base de $K \times K$, con $e_1=(1,0)$ y $e_2=(0,1)$. Si denotamos
      \[
      f(e_1) =
      \left( \begin{array}{c}
        a_{1,1} \\
        a_{2,1}
      \end{array}
      \right),\;
      f(e_2) =
      \left( \begin{array}{c}
        a_{1,2} \\
        a_{2,2}
      \end{array}
      \right) \in K \times K
      \]
      podemos escribir la imagen de cualquier elemento $x=(x_1,x_2) \in K \times K$ como sigue:
      \begin{align*}
          f(x) &= f\left(\begin{array}{c} x_1 \\ x_2 \end{array} \right) =
          f\left(\begin{array}{c} x_1 \\ 0 \end{array} \right) +
          f\left(\begin{array}{c} 0 \\ x_2 \end{array} \right) = x_1 f(e_1) + x_2 f(e_2) =\\
          &= x_1\left(\begin{array}{c} a_{1,1} \\ a_{1,2} \end{array} \right)
          x_2\left(\begin{array}{c} a_{1,2} \\ a_{2,2} \end{array} \right) =
          \left( \begin{array}{cc} a_{1,1} & a_{1,2} \\ a_{2,1} & a_{2,2}\end{array}\right)\left(\begin{array}{c} x_1 \\ x_2 \end{array} \right)
      \end{align*}

      Por tanto, podemos asociar cualquier endomorfismo de $K \times K$ con una matriz de dimensión 2 sobre $K$ ---y viceversa, pues el camino es biyectivo---, donde las columnas son las imágenes por el endomorfismo de la base de $K \times K$:
      \[
      f \longleftrightarrow \left( \begin{array}{cc} a_{1,1} & a_{1,2} \\ a_{2,1} & a_{2,2}\end{array}\right)
      \]

      Además, si $f$ tiene inversa, la matriz asociada es invertible y su inversa es la asociada a $f^{-1}$.

      Por tanto, concluimos que $Aut(C_p \times C_p)  \cong GL(p,2)$.
  \end{solucion}

  \begin{apartado}
      Para $p=2$ prueba que $Aut(C_2 \times C_2) \cong S_3$.
  \end{apartado}

  \begin{solucion}
      Es claro que cualquier $f \in Aut(C_2 \times C_2)$ tiene que dejar fijo al uno del cuerpo. Por tanto, si notamos $C_2 \times C_2 = \langle1,a,b \;|\; a^2=b^2=1, ab=ba\rangle = \{ 1,a,b,ab\}$, tenemos que todos los automorfismos que buscamos están determinados por las imágenes de los elementos ${a,b,ab}$. Estas pueden ser las siguientes:
      \begin{align*}
          a,b,ab \xrightarrow{id} a,b,ab \\
          a,b,ab \xrightarrow{f_1} a,ab,b \\
          a,b,ab \xrightarrow{f_2} b,a,ab \\
          a,b,ab \xrightarrow{f_3} b,ab,a \\
          a,b,ab \xrightarrow{f_4} ab,a,b \\
          a,b,ab \xrightarrow{f_5} ab,b,a
      \end{align*}

      Que los anteriores son automorfismos es claro por la construcción del cuerpo; como no hay más posibilidades, concluimos que
      \[
      Aut(C_2 \times C_2) = \{ 1,f_1,f_2,f_3,f_4,f_5 \}
      \]
      Sabemos, por otro lado, que $S_3$ es el grupo de permutaciones de un conjunto de tres elementos; esto es, el conjunto visto anteriormente. Es evidente entonces que
      \[
      Aut(C_2 \times C_2) \cong S_3
      \]

      Una forma quizás más sencilla de ver este resultado es el siguiente: por el apartado anterior, sabemos que $Aut(C_2 \times C_2) \cong GL(2,2)$; es decir, los automorfismos de $C_2 \times C_2$ son las matrices de orden 2 con elementos en $C_2$ que son invertibles; esto es, tales que su determinante es distinto de cero.

      Dada una matriz general
      \[
      \left(
      \begin{array}{cc}
          a & b \\
          c & d
      \end{array}
      \right)
      \]
      sabemos que es invertible si, y sólo si, $ad-bc \neq 0$. En nuestro caso, en el que los elementos están en $C_2 = \mathbb{Z}_2$, las posibles matrices son las siguientes:
      \begin{align*}
          1 = &\left(\begin{array}{cc} 1&0 \\ 0&1 \end{array}\right)
          \alpha = \left(\begin{array}{cc} 1&1 \\ 0&1 \end{array}\right)
          \beta = \left(\begin{array}{cc} 1&0 \\ 1&1 \end{array}\right) \\
          \\
          \gamma = &\left(\begin{array}{cc} 0&1 \\ 1&0 \end{array}\right)
          \delta = \left(\begin{array}{cc} 1&1 \\ 1&0 \end{array}\right)
          \delta^2 = \left(\begin{array}{cc} 0&1 \\ 1&1 \end{array}\right)
      \end{align*}

      Estamos entonces ante un grupo de orden 6, luego es $C_6$ ó $S_3$. Como hay más de un elemento de orden 2 ---de hecho, $\alpha^2 = \beta^2 = \gamma^2 = 1$---, el grupo tiene que ser $S_3$.
  \end{solucion}

  \begin{apartado}
      Determina el grupo $(C_2 \times C_2) \rtimes C_2$.
  \end{apartado}

  \begin{solucion}
      Para definir el grupo, lo único que necesitamos es la definición de la operación producto. En un producto semidirecto de grupos $N \rtimes H$, el producto se define como sigue:
      \[
      (n_1,h_1)(n_2,h_2) = (n_1\varphi(h_1)(n_2),h_1 h_2)
      \]
      donde $\varphi: H \to Aut(N)$ es un homomorfismo de grupos.

      En nuestro caso, $N = C_2 \times C_2$ y $H = C_2$. Como $Aut(N) = Aut(C_2 \times C_2) \cong S_3$, basta encontrar el homomorfismo de grupos $\varphi: C_2 \to S_3$. Como $C_2=\langle c | c^2=1\rangle = \{ 1,c \}$ sólo tiene un elemento distinto a la identidad, $c$, y este es de orden 2, los homomorfismos que buscamos ---además del trivial, que lleva todos los elementos a la identidad y en cuyo caso el producto semidirecto no es más que el producto directo--- serán aquellos que lleven $c$ a un elemento de orden 2 en $S_3$.

      Los elementos de orden 2 de $S_3$ son exactamente los ciclos de longitud 2; esto es: $(23), (12)$ y $(13)$. Con la notación del apartado anterior, estos ciclos son $f_1=(b\;\;ab), f_2=(a\;\;b)$ y $f_5=(a\;\;ab)$.

      Tenemos entonces las siguientes posibilidades:
      \begin{align*}
          \varphi_0 : C_2 &\longrightarrow Aut(C_2 \times C_2) \\
          1 &\longmapsto id \\
          c &\longmapsto id
      \end{align*}
      \begin{align*}
          \varphi_1 : C_2 &\longrightarrow Aut(C_2 \times C_2) \\
          1 &\longmapsto id \\
          c &\longmapsto (b\;\;ab)
      \end{align*}
      \begin{align*}
          \varphi_2 : C_2 &\longrightarrow Aut(C_2 \times C_2) \\
          1 &\longmapsto id \\
          c &\longmapsto (a\;\;b)
      \end{align*}
      \begin{align*}
          \varphi_3 : C_2 &\longrightarrow Aut(C_2 \times C_2) \\
          1 &\longmapsto id \\
          c &\longmapsto (a\;\;ab)
      \end{align*}

      Cabría esperar que los productos semidirectos asociados a cada $\varphi_i$ fueran diferentes pero, como veremos, tenemos esencialmente dos:
      \begin{align*}
          (C_2 \times C_2) \rtimes_{\varphi_0} C_2& = C_2 \times C_2 \times C_2 \\
          (C_2 \times C_2) \rtimes_{\varphi_1} C_2 = (C_2 \times C_2) \rtimes_{\varphi_2} C_2 = (C_2 \times C_2) \rtimes_{\varphi_3} C_2 &= D_4
      \end{align*}

      El primero es claro: $\varphi_0$ es el homomorfismo trivial que lleva todos los elementos a la identidad, luego el producto se define como
      \[
      (n_1,h_1)(n_2,h_2) = (n_1\varphi_0(h_1)(n_2),h_1 h_2) = (n_1 n_2,h_1 h_2)
      \]
      es decir, es el producto directo de los dos grupos con la operación usual.

      Que todos los demás homomorfismos generan el mismo grupo tampoco es difícil: en $S_3$, los elementos de orden 2 son todos conjugados los unos de los otros:
      \begin{align*}
          (12) = (23)(13)(23) = (13)(23)(13) \\
          (13) = (23)(12)(23) = (12)(23)(12) \\
          (23) = (13)(12)(13) = (12)(13)(12)
      \end{align*}
      Por tanto, al aplicarse sobre todos los elementos de $C_2$ generan el mismo grupo.

      Lo único que nos falta por ver es que el grupo es, de hecho, $D_4$. Estudiemos, por ejemplo, $(C_2 \times C_2) \rtimes_{\varphi_1} C_2$ ---como hemos visto, da igual estudiar este que cualquier de los otros dos---.

      Evidentemente, es un grupo de 8 elementos ---los elementos son todas las posibles parejas de un grupo de 4 elementos y otro de 2--- no abeliano. Para ver que no es abeliano basta dar dos elementos que no conmuten; consideramos, por ejemplo, $(a,c)$ y $(b,1)$:
      \begin{align*}
          (a,c)(b,1) = (a\varphi_1(c)(b),c1) = (aab, c) = (b,c) \\
          (b,1)(a,c) = (a\varphi_1(1)(a),1c) = (aa, c) = (1,c)
      \end{align*}

      Esto nos acota la búsqueda, pues grupos con esas características sólo hay dos: el diédrico, $D_4 = \{1,\sigma,\sigma^2,\sigma^3,\tau,\sigma\tau,\sigma^2\tau,\sigma^3\tau\}$, y el grupo de los cuaternios, $Q = \{1,-1,i,-i,j,-j,k,-k\}$.

      Sabemos que en el grupo de los cuaternios sólo hay un elemento de orden 2: el $-1$ ---todos los demás son de orden 4---. Pero en nuestro producto $(C_2 \times C_2) \rtimes_{\varphi_1} C_2$ podemos encontrar más de un elemento de orden 2; por ejemplo, $(1,c)$ y $(a,1)$:
      \begin{align*}
          (1,c)(1,c) = (1\varphi_1(c)(1),c^2) = (1,1) \\
          (a,1)(a,1) = (a\varphi_1(1)(a),1\cdot1) = (a^2,1) = (1,1)
      \end{align*}

      Por tanto, concluimos que:
      \[
      (C_2 \times C_2) \rtimes_{\varphi} C_2 = \begin{cases}
        C_2 \times C_2 \times C_2 &\textrm{ si } \varphi = \textrm{trivial} \\
        D4 &\textrm{ en otro caso}
      \end{cases}
      \]

  \end{solucion}

  \begin{apartado}
      Para $p=3$ calcular el grupo $Aut(C_3 \times C_3)$ y su estructura.
  \end{apartado}

  \begin{solucion}
      Por el segundo apartado del ejercicio, sabemos que $Aut(C_3 \times C_3) = GL(3,2)$; esto es, el grupo de las matrices invertibles de orden 2 con sus elementos en $C_3$.

      Un cálculo rápido en GAP nos permite ver el número de elementos que tiene este grupo:
      \begin{lstlisting}[language=GAP]
          gap> C3:=CyclicGroup(3);
          <pc group of size 3 with 1 generators>

          gap> C3xC3:=DirectProduct(C3,C3);
          <pc group of size 9 with 2 generators>

          gap> Size(Elements(AutomorphismGroup(C3xC3)));
          48
      \end{lstlisting}

      Esto nos da bastante información del grupo. Como $48 = 2^4*3 = 16*3$, el teorema de Sylow nos dice que el número $n$ de 3-subgrupos de Sylow ---que son de orden $3^1=3$--- debe cumplir:
      \begin{itemize}
          \item $n | 16$
          \item $n \equiv 1 (mod 3)$
      \end{itemize}
      Por tanto, podemos tener 1, 4, ó 16 3-subgrupos de Sylow. Resulta que hay 4 3-subgrupos de Sylow, pero lo que nos interesa es realmente otro aspecto:

      Sabemos que para un $p$ fijo, los $p$-subgrupos de Sylow son todos conjugados entre sí. Por tanto, tenemos que todos los 3-subgrupos de Sylow de $Aut(C_3 \times C_3)$ son conjugados. Como todos estos subgrupos son de orden 3, y todos los grupos de orden 3 son isomorfos al cíclico de orden 3 ---esto es, generados por un elemento de orden 3---, tenemos que \emph{todos} los elementos de orden 3 de $Aut(C_3 \times C_3)$ son conjugados entre sí. Esta propiedad nos ayudará a definir el producto semidirecto del último apartado con comodidad.
  \end{solucion}

  \begin{apartado}
      Determina un elemento de orden 3 de $Aut(C_3 \times C_3)$.
  \end{apartado}

  \begin{solucion}
      Para encontrar un elemento de orden 3 de $Aut(C_3 \times C_3)$ hay que buscar una matriz $2\times2$ en $C_3$ ---de aquí en adelante consideramos $C_3 = \mathbb{Z}_3$--- cuyo cubo sea la matriz identidad.

      Una opción para hacer esto es tomar una matriz $2\times2$ general
      \[
      \left(
      \begin{array}{cc}
          a & b \\
          c & d
      \end{array}
      \right)
      \]
      y calcular su cubo:
      \[
      A^3 = \left(
      \begin{array}{cc}
          b(ac+cd)+a(a^2+bc) & b(bc+d^2)+a(ab+bd) \\
          d(ac+cd)+c(a^2+bc) & d(bc+d^2)+c(ab+bd)
      \end{array}
      \right)
      \]

      Imponemos entonces $A^3 = I_2$ y tenemos un sistema no lineal de cuatro ecuaciones con cuatro incógnitas:
      \[
      \begin{cases}
           b(ac+cd)+a(a^2+bc) = 1 \\
           b(bc+d^2)+a(ab+bd) = 0 \\
           d(ac+cd)+c(a^2+bc) = 1 \\
           d(bc+d^2)+c(ab+bd) = 0
      \end{cases}
      \]

      Las soluciones para $a,b,c,d$ determinan todos los posibles elementos de orden 3.

      Podemos también buscar el elemento con un sistema de cálculo como GAP
      \begin{lstlisting}[language=GAP]
          gap> C3:=CyclicGroup(3);
          <pc group of size 3 with 1 generators>

          gap> C3xC3:=DirectProduct(C3,C3);
          <pc group of size 9 with 2 generators>

          gap> A:=AutomorphismGroup(C3xC3);
          <group with 4 generators>

          gap> Aut:=Elements(A);;

          gap> Filtered(Aut,x->Order(x)=3)[1];
          [ f1, f2 ] -> [ f1*f2^2, f2 ]
      \end{lstlisting}
  \end{solucion}

  El último comando simplemente filtra, de entre todos los elementos de $Aut(C_3 \times C_3)$, aquellos de orden 3 y devuelve el primero que se encuentra. La salida de este comando se interpreta como sigue: f1 y f2 son los dos generadores de $C_3 \times C_3$. Si lo vemos como $\mathbb{Z}_3 \times \mathbb{Z}_3$, podemos considerar $f1 = (1,0)$ y $f2 = (0,1)$.

  Entonces, el elemento que muestra la línea 13 del código anterior es el que lleva el $(1,0)$ en $(1,0)+(0,1)+(0,1) = (1,2)$ ---estamos escribiendo aquí en notación aditiva lo que GAP nos proporciona con notación multiplicativa--- y el $(0,1)$ en $(0,1)$; es decir, es la matriz
  \[
  B = \left(
  \begin{array}{cc}
      1 & 0 \\
      2 & 1
  \end{array}
  \right)
  \]
  que es un elemento de orden 3:
  \[
  B^3 = \left(
  \begin{array}{cc}
      1 & 0 \\
      6 & 1
  \end{array}
  \right) =
  \left(
  \begin{array}{cc}
      1 & 0 \\
      0 & 1
  \end{array}
  \right) = I_2
  \]

  \begin{apartado}
      Determina el grupo  $(C_3 \times C_3) \rtimes C_3$.
  \end{apartado}

  \begin{solucion}
      Como en el apartado 1.3, basta definir el homomorfismo $\varphi: C_3 \to Aut(C_3 \times C_3)$.

      Pero para determinar este homomorfismo basta determinar la imagen del generador de $C_3 = \mathbb{Z}_3$, al que llamamos $1$ ---usamos notación aditiva---. Como 1 es de orden 3 ---$1+1+1=3 \equiv 0 (mod 3)$---, su imagen por $\varphi$ tiene que ser un elemento de orden 3.

      Pero en el apartado 1.4 probamos que todos los elementos de orden 3 de $Aut(C_3 \times C_3)$ son conjugados, lo que nos permite afirmar que el producto semidirecto es independiente del elemento de orden 3 que tomemos como imagen del 1.

      Por tanto, construimos el homomorfismo con el elemento de orden 3 que obtuvimos en el apartado anterior:
      \begin{align*}
          \varphi : C_3 &\longrightarrow Aut(C_3 \times C_3) \\
          0 &\longmapsto id \\
          1 &\longmapsto B
      \end{align*}
      donde B es la matriz definida anteriormente.

      Tenemos entonces que $(C_3 \times C_3) \rtimes_{\varphi} C_3$ es un grupo no abeliano de orden 27 ---parejas de un grupo de orden 9 y de un grupo de orden 3--- con producto como sigue:
      \begin{align*}
          ((x_1,y_1),z_1)((x_2,y_2),z_2) &= ((x_1,y_1)\varphi(z_1)(x_2,y_2),z_1 z_2) =\\
          &= \begin{cases}
            ((x_1 x_2,y_1 y_2),0) &\textrm{ si } z_1=0 \\
            ((x_1,y_1)B(x_2,y_2),z_2) &\textrm{ si } z_1=1 \\
            ((x_1,y_1)B^2(x_2,y_2),2z_2) &\textrm{ si } z_1=2
        \end{cases}\\
        &= \begin{cases}
          ((x_1 x_2,y_1 y_2),0) &\textrm{ si } z_1=0 \\
          ((x_1,y_1)(x_2,2x_2+y_2),z_2) &\textrm{ si } z_1=1 \\
          ((x_1,y_1)(x_2,x_2+y_2),2z_2) &\textrm{ si } z_1=2
        \end{cases}\\
        &= \begin{cases}
          ((x_1 x_2,y_1 y_2),0) &\textrm{ si } z_1=0 \\
          ((x_1 x_2,y_1(2x_2+y_2)),z_2) &\textrm{ si } z_1=1 \\
          ((x_1 x_2,y_1(x_2+y_2)),2z_2) &\textrm{ si } z_1=2
        \end{cases}
    \end{align*}
    donde el producto de $B$ ---resp. $B^2$--- con el elemento $(x_2,y_2)$ es el producto matricial usual:
    \begin{align*}
        B(x_2,y_2) &= \left(
            \begin{array}{cc}
                1 & 0 \\
                2 & 1
            \end{array}
            \right) \left(
            \begin{array}{c}
                x_2 \\
                y_2
            \end{array}
            \right) = \left(
            \begin{array}{c}
                x_2 \\
                2x_2+y_2
            \end{array}
            \right) \\
        B^2(x_2,y_2) &= \left(
            \begin{array}{cc}
                1 & 0 \\
                1 & 1
            \end{array}
            \right) \left(
            \begin{array}{c}
                x_2 \\
                y_2
            \end{array}
            \right) = \left(
            \begin{array}{c}
                x_2 \\
                x_2+y_2
            \end{array}
            \right)
    \end{align*}

    Que es no abeliano es claro; basta tomar, por ejemplo, los elementos $((0,1),2)$ y $((1,1),2)$, que no conmutan:
    \begin{align*}
        ((0,1),2)((1,1),2) = ((0,1(1+1)),2\cdot2) = ((0,2),1) \\
        ((1,1),2)((0,1),2) = ((0,1(0+1)),2\cdot2) = ((0,1),1) \\
    \end{align*}

    Una búsqueda rápida en la \fnurl{wiki de GroupProps}{http://groupprops.subwiki.org/wiki/Groups_of_order_27} nos permite saber que hay sólo dos grupos no abelianos de orden 27: $M_{27} = C_9 \rtimes C_3$, \fnurl{el producto semidirecto del cíclico de orden nueve y del cíclico de orden tres}{http://groupprops.subwiki.org/wiki/M27}, y $UT(3,3)$, \fnurl{el grupo de matrices sobre $\mathbb{Z}_3$ unitriangulares}{{http://groupprops.subwiki.org/wiki/Prime-cube_order_group:U(3,3)}} ---es decir, matrices con todos los elementos de su diagonal igual a 1, todos los elementos debajo de la diagonal igual a 0 y elementos arbitrarios encima---,  de dimensión 3.

    Como $C_3 \times C_3 \ncong C_9$, entonces $(C_3 \times C_3) \rtimes C_3 \ncong M_{27}$. Por tanto, podemos concluir que el grupo que buscamos es isomorfo al grupo de matrices unitriangulares de dimensión 3.

    Además, como antes, si consideramos el homomorfismo trivial
    \begin{align*}
        \psi : C_3 &\longrightarrow Aut(C_3 \times C_3) \\
        0 &\longmapsto id \\
        1 &\longmapsto id
    \end{align*}
    tenemos el producto directo:
    \[
    (C_3 \times C_3) \rtimes_{\psi} C_3 = C_3 \times C_3 \times C_3
    \]

    Por tanto, el producto semidirecto que buscamos queda totalmente determinado:
    \[
    (C_3 \times C_3) \rtimes_{\varphi} C_3 = \begin{cases}
      C_3 \times C_3 \times C_3 &\textrm{ si } \varphi = \textrm{trivial} \\
      UT(3,3) &\textrm{ en otro caso}
    \end{cases}
    \]
  \end{solucion}
\end{document}
