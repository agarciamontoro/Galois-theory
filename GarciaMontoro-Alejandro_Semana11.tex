\documentclass[a4paper, 11pt]{article}
\usepackage[semana=11]{estilo}

\begin{document}

  \maketitle

  \begin{ejercicio}
      Prueba que la extensión $\mathbb{Q}(\sqrt{2+\sqrt{2}})/\mathbb{Q}$ es cíclica de grado 4.
  \end{ejercicio}

  \begin{solucion}
      Sea $\beta = \sqrt{2}$. Manipulando esta expresión, llegamos a lo siguiente:
      \begin{align*}
          \beta^2 &= 2 \\
          \beta^2 - 2 &= 0
      \end{align*}
      de donde deducimos que $\beta$ es raíz del polinomio $P(X) = X^2 - 2 \in \mathbb{Q}[X]$.

      Por el Teorema de Lagrange, sabemos que $\mathbb{Q}(\beta)/\mathbb{Q}$ es una extensión cíclica de grado 2.

      Sea ahora $\beta = \sqrt{1+\beta}$. Manipulando de nuevo:
      \begin{align*}
          \alpha^2 &= 1 + \beta \\
          \alpha^2 - (1 + \beta) &= 0 \\
      \end{align*}
      de donde deducimos que $\alpha$ es raíz del polinomio $Q(X) = X^2 - a \in \mathbb{Q}(\beta)[X]$, donde $a = 1 + \sqrt{2} \in \mathbb{Q}(\beta)$.

      Por el Teorema de Lagrange, sabemos que $\mathbb{Q}(\alpha)/\mathbb{Q}(\beta)$ es una extensión cíclica de grado 2.

      Concluimos entonces que $\mathbb{Q}(\alpha)/\mathbb{Q}$ tiene grado 4, aunque quedaría por demostrar si es cíclica o tiene al grupo de Klein como grupo de Galois.

      En general, que una extensión tenga subextensiones cíclicas no implica que la extensión sea cíclica. Hagámoslo de otra manera entonces:

      Reescribiendo la relación anterior tenemos:
      \begin{align*}
          \alpha^2 &= 1 + \beta \\
          \alpha^2 - 1 &= \beta \\
          \alpha^4 -2\alpha^2 + 1 &= 2 \\
          \alpha^4 -2\alpha^2 - 1 &= 0 \\
      \end{align*}
      de donde deducimos que el polinomio irreducible ---se ve claramente por Eisenstein--- de $\alpha$ es $f(X) = X^4 - 2X^2 -1$.

      Las cuatro raíces de este polinomio son $\pm\alpha$ y $\pm\sqrt{2-\sqrt{2}}$.

      Si tomamos $\sigma$ un elemento de $Aut(\mathbb{Q}(\alpha)/\mathbb{Q})$ tal que $\sigma(\alpha) = \sqrt{2-\sqrt{2}}$, entonces $\sigma(\sqrt{2}) = -\sqrt{2}$, luego
      \begin{align*}
          \sigma(\sigma(\alpha)) &= \sigma(\sqrt{2-\sqrt{2}}) = \sigma\left(\frac{\sqrt{2}}{\alpha}\right) = \frac{\sigma(\sqrt{2})}{\sigma(\alpha)} = \frac{-\sqrt{2}}{\sqrt{2-\sqrt{2}}} = -\sqrt{2+\sqrt{2}}
      \end{align*}

      Es decir, el orden de $\sigma$ es mayor estricto que 2. Como este orden tiene que dividir al orden del grupo de Galois, que sabemos que es 4 por el estudio anterior, concluimos que el orden de $\sigma$ es exactamente 4.

      De entre los dos posibles grupos que teníamos, esta condición nos deja una sola posibilidad: el grupo cíclico.

      Por tanto, la extensión $\mathbb{Q}(\alpha)/\mathbb{Q}$ es cíclica de grado 4.

  \end{solucion}

  \begin{ejercicio}
      Se considera la sucesión $\{\alpha_n\}_n$, definida:
      \begin{align*}
          \alpha_0 &= 0 \\
          \alpha_{n+1} &=  \sqrt{2+\alpha_n} &\forall n\in\mathbb{N}
      \end{align*}

      Prueba que la extensión $\mathbb{Q}(\alpha_n)/\mathbb{Q}$ es cíclica de grado $2^n$.
  \end{ejercicio}

  \begin{solucion}
      Podemos probar por inducción que $\alpha_n = 2 Re(\xi_{2^{n+2}})$, donde $\xi_k$ es la raíz $k$-ésima de la unidad.

      En efecto, $\alpha_0 = 0$ y $\xi_{2^2} = \xi_4 = i$, luego $2 Re(\xi_4) = 2\cdot0 = 0$ y, por tanto, $\alpha_0 = 2 Re(\xi_4)$.

      Supuesto cierto para $n$, probémoslo para $n+1$. Es decir, queremos probar que
      \[
      \alpha_{n+1} = 2 Re(\xi_{2^{n+3}})
      \]

      dado que
      \[
      \alpha_n = 2 Re(\xi_{2^{n+2}})
      \]

      Pero esto es evidente, pues sabemos que $\xi_{2^n} = \xi_{2^{n+1}}^2$. En nuestro caso, para que salgan los índices que queremos, trabajaremos con la igualdad
      \[
      \xi_{2^{n+2}} = \xi_{2^{n+3}}^2
      \]

      Si llamamos $a+bi = \xi_{2^{n+2}}$ y $x+yi = \xi_{2^{n+3}}^2$, tenemos que
      \[
      a + bi = (x + yi)^2 = (x^2-y^2) + 2xyi
      \]

      De esta igualdad y del hecho de que las raíces de la unidad están en la circunferencia unidad, tenemos el siguiente sistema:
      \begin{align*}
          a &= x^2 - y^2 \\
          1 &= x^2 + y^2
      \end{align*}
      de donde podemos obtener cuánto vale la parte real de $\xi_{2^{n+3}}^2$:
      \[
      a + 1 = 2x^2 \implies x = \sqrt{\frac{a+1}{2}}
      \]

      Como sabemos que $a = Re(\xi_{2^{n+2}}) = \frac{\alpha_n}{2}$, tenemos que
      \[
      x = \sqrt{\frac{\frac{\alpha_n}{2}+1}{2}} = \sqrt{\frac{\alpha_n + 2}{4}} = \frac{\sqrt{2+\alpha_n}}{2} = \frac{\alpha_{n+1}}{2}
      \]

      Es decir, tenemos que $\alpha_{n+1} = 2x = 2Re(\xi_{2^{n+3}})$, tal y como queríamos demostrar.

      Una vez comprobada esta relación, podemos ver la relación de los cuerpos que se generan, y observamos la siguiente torre:
      \[
      \mathbb{Q} \subset \mathbb{Q}(\alpha_n) \subset \mathbb{Q}(\xi_{2^{n+2}}) = E
      \]
      donde $\mathbb{Q}(\alpha_n) = E \cap \mathbb{R}$ es el cuerpo fijo para la conjugación compleja.

      Estudiando los grados, es claro lo siguiente:
      \begin{itemize}
          \item $[\mathbb{Q}(\xi_{2^{n+2}}):\mathbb{Q}(\alpha_n)] = 2$
          \item $[\mathbb{Q}(\xi_{2^{n+2}}):\mathbb{Q}] = \varphi(2^{n+2}) = 2^{n+1}$
      \end{itemize}

      Por tanto, tenemos que la extensión que nosotros buscamos, $\mathbb{Q}(\alpha_n)/\mathbb{Q}$ tiene grado $\frac{2^{n+1}}{2} = 2^n$.

      Por otro lado, sabemos que $Gal(\mathbb{Q}(\xi_{2^{n+2}})) = \mathbb{Z}_{2^{n+2}}^\times$, el grupo multiplicativo de los enteros módulo $2^{n+2}$.

      Además, es conocido que $\mathbb{Z}_{2^{n+2}}^\times = C_2 \times C_{2^n}$, producto de grupos cíclicos.

      Por tanto, para calcular el grupo de Galois de la extensión $\mathbb{Q}(\alpha_n)/\mathbb{Q}$, basta observar que es el cociente entre los grupos de Galois de las extensiones que acabamos de estudiar; es decir:

      \[
      Gal(\mathbb{Q}(\alpha_n)/\mathbb{Q}) =  \frac{Gal(\mathbb{Q}(\xi_{2^{n+2}})/\mathbb{Q})}{Gal(\mathbb{Q}(\xi_{2^{n+2}})/\mathbb{Q}(\alpha_n))} = \frac{C_2 \times C_{2^n}}{C_2} = C_{2^n}
      \]

      Concluimos así que la extensión $\mathbb{Q}(\alpha_n)/\mathbb{Q}$ es cíclica de grado $2^n$, tal y como queríamos demostrar.
    \end{solucion}
\end{document}
