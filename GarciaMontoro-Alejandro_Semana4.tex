\documentclass[a4paper, 11pt]{article}
\usepackage[semana=4]{estilo}

\begin{document}

  \maketitle

  \begin{ejercicio}
      Sea $E/K$ una extensión algebraica tal que para cualquier extensión finita $F/K$ existe un K-homomorfismo $\sigma:F/K\to E/K$. Demuestra  que entonces E es una clausura algebraica de K.
  \end{ejercicio}

  \begin{solucion}
      Sea $E/K$ la extensión algebraica que se especifica en el enunciado. Queremos ver que $E$ es una clausura algebraica de $K$; es decir, que $E$ es una extensión algebraica y es un cuerpo algebraicamente cerrado sobre $K$. Usaremos una caracterización de clausura algebraica muy interesante para este ejercicio: el cuerpo $E$ es una clausura algebraica sobre $K$ si $E/K$ es una extensión algebraica y todo polinomio no constante $p\in K[X]$ descompone en $E[X]$.

      Lo primero lo tenemos por hipótesis. Para probar lo segundo, tomemos $p\in K[X]$ un polinomio cualquiera no constante con coeficientes en $K$.

      Consideramos $F$, el cuerpo de descomposición de $p$. Por la definición de cuerpo de descomposición sabemos que la extensión $F/K$ es finita\footnote{Como $F=K(\alpha_1,\dots,\alpha_n)$ ---donde $\alpha_i$ son todas las raíces de $p$ y $n$, su grado--- tenemos que $[F:K]\leq n!$}.

      Podemos entonces aplicar la hipótesis del enunciado: como $F/K$ es finita, existe un homomorfismo $\sigma : F/K \to E/K$. El polinomio $p$ se queda invariante por $\sigma$, pues los coeficientes de $p$ están en $K$ y estos elementos son fijos por $\sigma$. Además, tenemos que:
      \begin{align*}
          p &= \sigma(p) = \sigma\left((X-\alpha_1)\cdots(X-\alpha_n)\right) = (X-\sigma(\alpha_1))\cdots(X-\sigma(\alpha_n)) = \\
          &= (X-\beta_1)\cdots(X-\beta_n)
      \end{align*}
      donde $\beta_i = \sigma(\alpha_i) \in E$.

      De aquí deducimos directamente, junto con la hipótesis del enunciado, que $E$ cumple la caracterización expuesta anteriormente, con lo que $E$ es una clausura algebraica de $K$.
  \end{solucion}

  \newpage

  \begin{ejercicio}
      Demuestra que la clausura algebraica de $\mathbb{Q}$ no es una extensión finita de $\mathbb{Q}$.
  \end{ejercicio}

  \begin{solucion}
      Consideremos el polinomio $p_n = X^n-2$, donde $n\in\mathbb{N}$. Es claro que $q=2$ divide a todos los coeficientes menos al líder y que su cuadrado $q^2=4$ no divide al coeficiente constante, sea cual sea $n$. Por tanto, el criterio de Eisenstein nos dice que $p_n$ es irreducible, $\forall n \in \mathbb{N}$.

      Por otro lado, sabemos que existe una extensión $F_n/\mathbb{Q}$ en la que $p_n$ tiene al menos una raíz: sea ésta $\alpha$. Como $F_n=\mathbb{Q}[\alpha]$ y tenemos que $Irr(\alpha,\mathbb{Q}) = p_n$ tiene grado $n$, entonces $F_n/\mathbb{Q}$ es de grado $n$. Podemos crear entonces una extensión de grado $n$ por cada polinomio $p_n$.

      Si $\bar{\mathbb{Q}}$ es la clausura algebraica de $\mathbb{Q}$, tenemos la torre de cuerpos
      \[
      \mathbb{Q} \subset F_1 \subset F_2 \subset \cdots \subset F_n \subset \cdots \subset \bar{\mathbb{Q}}
      \]

      Entonces, $\bar{\mathbb{Q}}$ tiene subextensiones de cualquier grado. Concluimos entonces que $[\bar{\mathbb{Q}}:\mathbb{Q}] = \infty$.
  \end{solucion}

  \begin{ejercicio}
      Sea $F/K$ una extensión con F algebraicamente cerrado. Si llamamos
      \[
      E = \{\alpha\in F \;|\; \text{$\alpha$ es algebraico sobre $K$}\}
      \]
      demuestra que $E$ es una clausura algebraica de K.
  \end{ejercicio}

  \begin{solucion}
      Tenemos que demostrar que $E$ es una clausura algebraica de $K$; es decir, que $E/K$ es una extensión algebraica y que $E$ es un cuerpo algebraicamente cerrado.

      Lo primero lo tenemos por la definición de $E$. Para probar lo segundo, basta con demostrar una de las caracterizaciones equivalentes de cuerpo algebraicamente cerrado; a saber:
      \begin{enumerate}
          \item \label{res} Todo polinomio no constante en $E[X]$ tiene al menos una raíz en $E$.
          \item Todo polinomio no constante en $E[X]$ descompone en $E$.
          \item Todo polinomio no constante irreducible en $E[X]$ tiene grado 1.
          \item $E$ no tiene extensiones algebraicas propias.
      \end{enumerate}

      Vamos a usar la caracterización \ref{res}, así que tomamos
      \[
      p(X) = \sum_{i=0}^n a_n X^i \in E[X]
      \]
      un polinomio cualquiera no constante con coeficientes en $E$.

      Como $F$ es algebraicamente cerrado y $E\subset F$ sabemos, por \ref{res}, que existe en F una raíz de $p$; esto es, $\exists\alpha\in F \;/\; p(\alpha)=0$.

      Podemos considerar entonces $K'=K(a_0,\dots,a_n)$ el cuerpo extensión de $K$ con todos los coeficientes del polinomio $p$. Es evidente por la construcción de $K'$ que $\alpha$ es algebraico sobre $K'$.

      Pero por hipótesis los $a_i$, por ser elementos de $E$, son algebraicos sobre $K$, luego $\alpha$ es algebraico también sobre $K$ y, por tanto, cumple la condición dada por la definición de $E$. Esto es, $\alpha\in E$.

      Tenemos entonces que para todo polinomio no constante en $E[X]$, existe una raíz en $E$; es decir, $E$ es algebraicamente cerrado.

      Concluimos así que $E$ es una clausura algebraica de K.
  \end{solucion}
\end{document}
